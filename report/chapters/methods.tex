\chapter{Methods}\label{chap:methods}
\section{Task 1: Configuration of eBPF/XDP }
In this section, we describe the steps to download, configure, and use the xdp-tools from the XDP project. The tools we focus on are xdp-filter and xdp-dump.

\subsection{Downloading and Configuring xdp-tools}
To begin, navigate to the xdp-tools repository on GitHub at \url{https://github.com/xdp-project/xdp-tools}. Clone the repository using the following command:
\begin{verbatim}
git clone https://github.com/xdp-project/xdp-tools.git
\end{verbatim}

Next, navigate to the cloned directory and follow the instructions in the README file to build and install the tools:
\begin{verbatim}
cd xdp-tools
make
sudo make install
\end{verbatim}

\subsection{xdp-filter}
The xdp-filter tool allows you to filter packets based on various criteria. To use xdp-filter, navigate to the xdp-filter directory:
\begin{verbatim}
cd xdp-tools/xdp-filter
\end{verbatim}

You can run example commands to filter packets. For instance, to filter packets based on a specific IP address, use:
\begin{verbatim}
sudo ./xdp-filter --ip 192.168.1.1
\end{verbatim}

\begin{figure}[h]
\centering
\includegraphics[width=0.8\textwidth]{../images/uni-logo.png}
\caption{Example output of xdp-filter filtering packets by IP address}
\label{fig:xdp-filter-example}
\end{figure}

\subsection{xdp-dump}
The xdp-dump tool is used to dump packet data for analysis. Navigate to the xdp-dump directory:
\begin{verbatim}
cd xdp-tools/xdp-dump
\end{verbatim}

To dump packets, you can use the following command:
\begin{verbatim}
sudo ./xdp-dump
\end{verbatim}

\begin{figure}[h]
\centering
\includegraphics[width=0.8\textwidth]{../images/uni-logo.png}
\caption{Example output of xdp-dump showing dumped packet data}
\label{fig:xdp-dump-example}
\end{figure}

These tools provide powerful capabilities for filtering and analyzing network packets using XDP.

